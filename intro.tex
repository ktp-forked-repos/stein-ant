\chapter{Introduction}

\section{Mathematical background}
In addition to general mathematical maturity, 
this book assumes you have the following background:
\begin{itemize}\setlength{\itemsep}{-.7ex}
\item Basics of finite group theory
\item Commutative rings, ideals, quotient rings
\item Some elementary number theory
\item Basic Galois theory of fields
\item Point set topology
\item Basic of topological rings, groups, and measure theory
\end{itemize}
For example, if you have never worked with finite groups before, you
should read another book first. If you haven't seen much elementary
ring theory, there is still hope, but you will have to do some
additional reading and exercises.  We will briefly review the basics of
the Galois theory of number fields.

Some of the homework problems involve using a computer, but there
are examples which you can build on.  We will not assume that you have
a programming background or know much about algorithms. Most
of the book uses Sage {\tt http://sagemath.org}, which is
free open source mathematical software.  The following is an example
Sage session:
\begin{verbatim}
sage: 2 + 2
4 
sage: k.<a> = NumberField(x^2 + 1); k
Number Field in a with defining polynomial x^2 + 1
\end{verbatim}

\section{What is algebraic number theory?}
A number field $K$ is a finite degree algebraic extension of the rational
numbers $\Q$.  The primitive element theorem from Galois theory
asserts that every such extension can be represented as the set of all
polynomials of degree at most $d = [K:\Q] = \dim_{\Q} K$ 
in a single algebraic number $\alpha$:
$$
 K = \Q(\alpha) = \left\{ \sum_{n=0}^{m} a_n \alpha^n : a_n\in\Q \right\}.
$$ 
Here $\alpha$ is a root of a polynomial with coefficients in $\Q$.
\begin{comment}
Note that
$\Q(\alpha)$ is non-canonically isomorphic to $\Q[x]/(f)$, where $f$
is the minimal polynomial of~$\alpha$.  The isomorphism is induced by
the homomorphism $\Q[x]\to\Q(\alpha)$ that sends~$x$ to~$\alpha$,
which has kernel~$(f)$.  It is not canonical, since $\Q(\alpha)$ could
have nontrivial automorphisms.  For example, if $\alpha=\sqrt{2}$, then
$\Q(\sqrt{2})$ is isomorphic as a field to $\Q(-\sqrt{2})$ via
$\sqrt{2}\mapsto -\sqrt{2}$.  There are two isomorphisms
$\Q[x]/(x^2-2)\to \Q(\sqrt{2})$.
\end{comment}

\defn{Algebraic number theory} involves using techniques from (mostly
commutative) algebra and finite group theory to gain a deeper
understanding of the arithmetic of number fields and related objects
(e.g., functions fields, elliptic curves, etc.).  The main objects that we
study in this book are number fields, rings of integers
of number fields, unit groups, ideal class groups, norms, traces,
discriminants, prime ideals, Hilbert and other class fields and
associated reciprocity laws, zeta and $L$-functions, and algorithms
for computing each of the above.

\subsection{Topics in this book}
These are some of the main topics that are discussed in this book:
\begin{itemize}\setlength{\itemsep}{-.7ex}
\item Rings of integers of number fields
\item Unique factorization of ideals in Dedekind domains
\item Structure of the group of units of the ring of integers
\item Finiteness of the group of equivalence classes
of ideals of the ring of integers (the ``class group'')
\item Decomposition and inertia groups, Frobenius elements
\item Ramification
\item Discriminant and different
\item Quadratic and biquadratic fields
\item Cyclotomic fields (and applications)
\item How to use a computer to compute with many of the above 
objects (both algorithms and actual use of software).
\item Valuations on fields
\item Completions ($p$-adic fields)
\item Adeles and Ideles
\end{itemize}
Note that we will not do anything nontrivial with zeta functions or
$L$-functions. \edit{This is for another course -- add this to the
book later.}



\section{Some applications of algebraic number theory}
The following examples illustrate that learning algebraic number
theory as soon as possible is an excellent investment of your time.

\begin{enumerate}
\item {\bf Integer factorization} using the number field sieve.  The
number field sieve is the asymptotically fastest known algorithm for
factoring general large integers (that don't have too special of a
form).  Recently, in December 2003, the number field sieve was used to
factor the RSA-576 \$10000 challenge:
$$
\begin{array}{l}
1881988129206079638386972394616504398071635633794173827007\ldots\\
\ldots6335642298885971523466548531906060650474304531738801130339\ldots\\
\ldots6716199692321205734031879550656996221305168759307650257059\\
=39807508642406493739712550055038649119906436234252670840\ldots\\
\hspace{1em}\ldots6385189575946388957261768583317\\
\hspace{2em}\times
47277214610743530253622307197304822463291469530209711\ldots\\
\hspace{3em}\ldots6459852171130520711256363590397527
\end{array}
$$
(The $\ldots$ indicates that the newline should be removed, not that
there are missing digits.)

\item {\bf Primality test:} Agrawal and his students Saxena and Kayal from
India found in 2002 the first ever deterministic
polynomial-time (in the number of digits) primality test.  There
methods involve arithmetic in quotients of $(\Z/n\Z)[x]$, which are
best understood in the context of algebraic number theory.  For
example, Lenstra, Bernstein, and others have done that and improved
the algorithm significantly.

\item {\bf Deeper point of view} on questions in number theory:
\begin{enumerate}
\item Pell's Equation ($x^2-dy^2=1$) $\Longrightarrow$ Units in real quadratic fields $\Longrightarrow$ Unit groups in number fields
\item Diophantine Equations $\Longrightarrow$ For which $n$ does $x^n+y^n=z^n$ have a 
nontrivial solution?
\item Integer Factorization $\Longrightarrow$ Factorization of ideals
\item Riemann Hypothesis $\Longrightarrow$ Generalized Riemann Hypothesis
\item Deeper proof of Gauss's quadratic reciprocity law in terms of arithmetic
of cyclotomic fields $\Q(e^{2\pi i/n})$, which leads to class field theory.
\end{enumerate}
\item Wiles's proof of {\bf Fermat's Last Theorem}, i.e., that the
equation $x^n+y^n=z^n$ has no solutions with $x,y,z,n$ all positive
integers and $n\geq 3$, uses methods from
algebraic number theory extensively, in addition to many other deep
techniques.  Attempts to prove Fermat's Last Theorem long ago were
hugely influential in the development of algebraic number theory
by Dedekind, Hilbert, Kummer, Kronecker, and others.
\item {\bf Arithmetic geometry:} This is a huge field that studies
solutions to polynomial equations that lie in arithmetically
interesting rings, such as the integers or number fields.  A famous
major triumph of arithmetic geometry is Faltings's proof of Mordell's
Conjecture.
\begin{theorem}[Faltings] \label{thm:faltings}\ithm{Faltings}
Let $X$ be a nonsingular plane algebraic curve over a number
field $K$.  Assume that the manifold $X(\C)$ of complex solutions to
$X$ has genus at least $2$ (i.e., $X(\C)$ is topologically a donut
with two holes).  Then the set $X(K)$ of points on $X$ with
coordinates in~$K$ is finite.
\end{theorem}  
For example, Theorem~\ref{thm:faltings} implies that for any $n\geq 4$
and any number field~$K$, there are only finitely many solutions
in~$K$ to $x^n+y^n=1$.  

A major open problem in arithmetic geometry is the {\em Birch
  and Swinnerton-Dyer conjecture}. 
An \defn{elliptic curves} $E$ is an algebraic curve with at least one point
with coordinates in $K$ such that the set of complex points
$E(\C)$ is a topological torus.
The Birch and Swinnerton-Dyer conjecture gives a
criterion for whether or not $E(K)$ is infinite in
terms of analytic properties of the $L$-function $L(E,s)$.

\end{enumerate}


%%% Local Variables: 
%%% mode: latex
%%% TeX-master: "ant"
%%% End: 
